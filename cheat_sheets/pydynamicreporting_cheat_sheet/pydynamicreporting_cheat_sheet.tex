\documentclass[9pt,landscape]{article}
\usepackage{{./../static/style}}
\pdfinfo{
  /Title (Pydynamicreporting Cheat Sheet)
  /Creator (TeX)
  /Producer (pdfTeX 1.40.0)
  /Author (Ansys)
  /Subject (Pydynamicreporting)
  /Keywords (Pydynamicreporting, Cheat sheet)}

\begin{document}
\raggedright
\footnotesize

% Add the title of cheat sheet here
% ----------------------------------------

\begin{center}
     \Huge{\textbf{PyDynamicReporting Cheat sheet}} \\
\end{center}
\begin{center}
  \small{\textbf{version:0.4.0 (stable)}} \\
\end{center}
\AddToShipoutPicture*
  {\put(670,577.5){\includegraphics[height = 1.2cm]{ansys.png}}}
\AddToShipoutPictureBG*{\includegraphics[width=\paperwidth]{bground.png}}
\vspace{-0.15cm}
\noindent\makebox[\linewidth]{\rule{\paperwidth}{2pt}}

\begin{multicols}{3}
\setlength{\premulticols}{1pt}
\setlength{\postmulticols}{1pt}
\setlength{\multicolsep}{1pt}
\setlength{\columnsep}{2pt}

\section{\includegraphics[height=\fontcharht\font`\S]{slash.png} ADR Service}
To launch and stop a local ADR Service on a new database.
\begin{lstlisting}[language=Python]
import ansys.dynamicreporting.core as adr
db_dir = r'C:\tmp\my_local_db_directory'
ansys_ins = r'C:\Program Files\Ansys Inc\v241'
adr_service = adr.Service(
  ansys_installation=ansys_ins, db_directory=db_dir)
session_guid = adr_service.start(create_db=True)
# To stop the service
adr_service.stop()
\end{lstlisting}

To connect to an already running ADR service on port 8000.

\begin{lstlisting}[language=Python]
import ansys.dynamicreporting.core as adr
ansys_ins = r'C:\Program Files\Ansys Inc\v241'
adr_service = adr.Service(
  ansys_installation=ansys_ins)
adr_service.connect(url='http://localhost:8000') 
\end{lstlisting}

\section{\includegraphics[height=\fontcharht\font`\S]{slash.png}  Create new items}
To create new items in the database of different types
\begin{lstlisting}[language=Python]
# Create text item
my_text = adr_service.create_item(obj_name='Text')
my_text.item_text = "<h1>Simple Title</h1>Abc..."
# Create table item
import numpy as np
my_table = adr_service.create_item(obj_name='Table')
my_table.table_dict["rowlbls"] = ["Row 1", "Row 2"]
my_table.item_table = np.array([
  ["1", "2", "3", "4", "5"], 
  ["1", "4", "9", "16", "25"]], dtype="|S20")
# Create image item
img = adr_service.create_item(obj_name='Image')
img.item_image = r'C:\tmp\test_image.png'
# Create 3D item
scene = adr_service.create_item(obj_name='3D Scene')
scene.item_scene = r'C:\tmp\test_scene.avz'
# Create a tree item via a dictionary
leaves = []
for i in range(5):
  leaves.append({"key": "leaves", "name": f"Leaf {i}",
    "value": i})
children = []
children.append({"key": "child", "name": "Boolean example", "value": True})
children.append({
  "key": "child_parent", 
  "name": "A child parent", 
  "value": "Parents can have values", 
  "children": leaves,
  "state": "collapsed"})
tree = []
tree.append(
  {"key": "root", "name": "Top Level", "value": None, 
  "children": children, "state": "expanded"})
my_tree = adr_service.create_item(obj_name="Tree")
my_tree.item_tree = tree
\end{lstlisting} 


\section{\includegraphics[height=\fontcharht\font`\S]{slash.png}  Set plot properties}
On the item you can set properties that will modify how they are rendered. 
This is particularly important to control the visualization of a plot, which comes 
from a table item. Some typical settings are presented here
\begin{lstlisting}[language=Python]
# Set visualization to be plot instead of table
my_table.plot = 'line'
# Set X axis and axis formatting
my_table.xaxis = 'Row 1'
my_table.format = 'floatdot1'
\end{lstlisting} 

\section{\includegraphics[height=\fontcharht\font`\S]{slash.png}  Tag items}
Set tags for the items

\begin{lstlisting}[language=Python]
my_text.set_tags("tag1=one tag2=two tag3=three")
\end{lstlisting} 

Add or remove tags on an item
\begin{lstlisting}[language=Python]
my_text.add_tag(tag='tag4', value='four')
my_text.rem_tag("tag1")
\end{lstlisting} 

\section{\includegraphics[height=\fontcharht\font`\S]{slash.png} Create report templates}
The creation of report templates follows the ADR low level API. The first step is to define 
the server object
\begin{lstlisting}[language=Python]
server = adr_service.serverobj
\end{lstlisting}
Then use the low level API to create a new report template
\begin{lstlisting}[language=Python]
template_0=server.create_template(name="My Report", 
  parent=None, report_type="Layout:basic")
server.put_objects(template_0)

template_1=server.create_template(name="Intro", 
  parent=template_0, report_type="Layout:panel")
template_1.set_filter("A|i_type|cont|html,string;")
server.put_objects(template_0)
server.put_objects(template_1)

template_2=server.create_template(name="Plot", 
  parent=template_0, report_type="Layout:panel")
template_2.set_filter("A|i_type|cont|table;")
server.put_objects(template_2)
server.put_objects(template_0)
\end{lstlisting}

\section{\includegraphics[height=\fontcharht\font`\S]{slash.png}  Visualize items and reports}
Query the database for the items
\begin{lstlisting}[language=Python]
all_items = adr_service.query()
only_text_items = adr_service.query(filter=
  "A|i_type|cont|html,string")
\end{lstlisting}
Visualize an item
\begin{lstlisting}[language=Python]
only_text_items[0].visualize()
\end{lstlisting}

Visualize the default report, which corresponds to all the items in the report listed one after the other
\begin{lstlisting}[language=Python]
adr_service.visualize_report()
\end{lstlisting}
Get all reports in the database by name
\begin{lstlisting}[language=Python]
all_reports = adr_service.get_list_reports()
\end{lstlisting}
Find and visualize a report with a specific name
\begin{lstlisting}[language=Python]
report_by_name = adr_service.get_report(
  report_name='My Report')
report_by_name.visualize()
\end{lstlisting}


\section{\includegraphics[height=\fontcharht\font`\S]{slash.png}  Get url for items and reports}
Get the url corresponding to an item. It is an object property
\begin{lstlisting}[language=Python]
only_text_items[0].url
\end{lstlisting}
Get the url corresponding to a report. It is a class method
\begin{lstlisting}[language=Python]
report_by_name.get_url()
\end{lstlisting}


% Add subsection
% This section includes useful links to the documentation.
% Examples: installation, API reference, commands, examples.
% Replace 'name of link' with appropriate display text.

\subsection{References from PyAnsys Documentation}
\begin{itemize}
\item \href{https://dynamicreporting.docs.pyansys.com/version/stable/gettingstarted/index.html}{\color{blue}{Getting Started}}
\item \href{https://dynamicreporting.docs.pyansys.com/version/stable/class_documentation.html}{\color{blue}{API reference}}}
\item \href{https://dynamicreporting.docs.pyansys.com/version/stable/examples/index.html}{\color{blue}{Examples}}}
\end{itemize}
\end{multicols}

% Footer session of the latex with link to documentation and GitHub page
\vspace{-0.15cm}
\noindent\makebox[\linewidth]{\rule{\paperwidth}{4pt}}
\begin{center}
Getting Started with PyAnsys \includegraphics[height=\fontcharht\font`\S]{slash.png} \href{https://github.com/pyansys}{\color{blue}{PyAnsys on GitHub}} \includegraphics[height=\fontcharht\font`\S]{slash.png} Visit \code{\href{https://dev.docs.pyansys.com/}}{\color{blue}{dev.docs.pyansys.com}}
\end{center}
\end{document}