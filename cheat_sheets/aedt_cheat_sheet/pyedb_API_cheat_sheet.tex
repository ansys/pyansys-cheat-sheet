\documentclass[landscape]{article}
\usepackage{{./../static/style}}

\pdfinfo{
  /Title (PyAnsys Cheatsheets)
  /Creator (TeX)
  /Producer (pdfTeX 0.40.0)
  /Author (Ansys Inc.,)
  /Subject (PyAEDT)
  /Keywords (PyAEDT,AEDT,EDB)}

% -----------------------------------------------------------------------

\begin{document}
\raggedright
\footnotesize
\begin{center}
     \Huge{\textbf{PyAEDT EDB-API Cheatsheet}} \\
\end{center}
\AddToShipoutPicture*
	{\put(670,577.5){\includegraphics[height = 1.2cm]{ansys.png}}}
\AddToShipoutPictureBG*{\includegraphics[width=\paperwidth]{bground.png}}
\vspace{-0.15cm}
\noindent\makebox[\linewidth]{\rule{\paperwidth}{2pt}}

\begin{multicols}{3}
% multicol parameters
% These lengths are set only within the two main columns
%\setlength{\columnseprule}{0.25pt}
\setlength{\premulticols}{1pt}
\setlength{\postmulticols}{1pt}
\setlength{\multicolsep}{1pt}
\setlength{\columnsep}{2pt}

\section{\includegraphics[height=\fontcharht\font`\S]{slash.png} Launching EDB-API using PyAEDT}
\justifying
EDB manager manages the AEDB Database. An AEDB database is a folder that contains the database representing any part of a PCB. It can be opened and edited using the Edb class.
\begin{lstlisting}[language=Python]
# To launch an instance
import pyaedt
edb = pyaedt.Edb(edbversion="2023.1", edbpath= aedb_path)
edb.save_edb() # Save the edb file
edb.close_edb() # exits the edb file
\end{lstlisting}

\section{\includegraphics[height=\fontcharht\font`\S]{slash.png} Stackup and Layers}
These classes are the containers of the layer and stackup manager of the EDB API.
\begin{lstlisting}[language=Python]
# Adding a stackup layer
edb.stakup.add_layer("Name")
# To get the names of the all layers
edb.stackup.stackup_layers.keys()
\end{lstlisting}
\section{\includegraphics[height=\fontcharht\font`\S]{slash.png} Modeler and primitives}
These classes are the containers of primitives and all relative methods. Primitives are planes, lines, rectangles, and circles.
\begin{lstlisting}[language=Python]
# Creating a polygon by defining points
points = [[0.0, 1e-3], [0.0, 10e-3], [100e-3, 10e-3], [100e-3, 1e-3], [0.0, 1e-3]]
edb.modeler.create_polygon_from_points(points, layer_name = "Name")
\end{lstlisting}
\section{\includegraphics[height=\fontcharht\font`\S]{slash.png} Components}
Component class contains API references for net management. The main component object is called directly from main application using the property components.
\begin{lstlisting}[language=Python]
# To get the list of components
edb.components.components.keys()
# To get the net information of a component
edb.components.get_component_net_connection_info("Q13N")
\end{lstlisting}
\section{\includegraphics[height=\fontcharht\font`\S]{slash.png} Nets}
The Net class contains API references for net management. The main net object is called directly from main application using the property nets.
\begin{lstlisting}[language=Python]
# List of all nets in available in AEDB file
edb.nets.netlist
# To delete a net
edb.nets["net_name"].delete()
\end{lstlisting}
\section{\includegraphics[height=\fontcharht\font`\S]{slash.png} Vias and padstacks}
These containers contains the API references for padstack management. The main padstack object is called directly from main application using the property padstacks.
%# edb.padstacks.create_padstack(padstackname="p_name", holediam="$via_hole_size", antipaddiam="$antipaddiam", paddiam="$paddiam")
\begin{lstlisting}[language=Python]
# Creating a via
edb.padstacks.place(position = [5e-3, 5e-3], "MyVia")
# To get the pad parameters
edb.padstacks.get_pad_parameters()
\end{lstlisting}

\section{\includegraphics[height=\fontcharht\font`\S]{slash.png} Sources and Excitation}
These classes are the containers of sources methods of the EDB for both HFSS and Siwave
% edb.hfss.create_differential_wave_port(trace_p[0].id, ["0.0", "($ms_width+$ms_spacing)/2"], trace_n[0].id, ["0.0", "-($ms_width+$ms_spacing)/2"], )

\begin{lstlisting}[language=Python]
# To get the Dictionary of EDB excitations
edb.excitations
# To create differential port
edb.hfss.create_differential_wave_port(positive_primitive_id = trace_p[0].id, positive_points_on_edge = p1_points, negative_primitive_id = trace_n[0].id, negative_points_on_edge = n1_points, name = "wave_port_1")


\end{lstlisting}

\section{\includegraphics[height=\fontcharht\font`\S]{slash.png} Simulation Setup}
These classes are the containers of setup classes in EDB for both HFSS and Siwave.
\begin{lstlisting}[language=Python]
# HFSS simulation setup
setup = edb.create_hfss_setup(name = "my_setup")
setup.set_solution_single_frequency()
setup.hfss_solver_settings.enhanced_low_freq_accuracy = True
setup.hfss_solver_settings.order_basis = "first"
setup.adaptive_settings.add_adaptive_frequency_data("5GHz",8,"0.01")
\end{lstlisting}

\begin{lstlisting}[language=Python]
# SiWave Simulation setup
setup = edb.siwave.add_siwave_dc_analysis(name = "myDCIR_4")
setup.use_dc_custom_settings = True
setup.dc_slider_position = 0
setup.add_source_terminal_to_ground("V1", 1)
solve_edb = edb.solve_siwave()
\end{lstlisting}
\section{\includegraphics[height=\fontcharht\font`\S]{slash.png} Simulation Configuration}
These classes are the containers of simulation configuration constructors for the EDB.
\begin{lstlisting}[language=Python]
# AC settings
Sim_setup.ac_settings.start_freq = "100Hz"
sim_setup.ac_settings.stop_freq = "6GHz"
sim_setup.ac_settings.step_freq = "10MHz"

# Batch solve
sim_setup=edbapp.new_simulation_configuration()
sim_setup.solver_type sim_setup.SOLVER_TYPE.SiwaveSYZ
sim_setup.batch_solve_settings.cutout_subdesign_expansion = 0.01
sim_setup.batch_solve_settings.do_cutout_subdesign = True
sim_setup.use_default_cutout = False
sim_setup.batch_solve_settings.signal_nets = singl_net_list
sim_setup.batch_solve_settings.components = component_list
sim_setup.batch_solve_settings.power_nets = power_nets_list

# Saving config file
sim_setup.export_json(os.path.join(project_path, "configuration.json"))
edbapp.build_simulation_project(sim_setup)
\end{lstlisting}
\section{\includegraphics[height=\fontcharht\font`\S]{slash.png} SiWave Manger}
Siwave is a specialized tool for power integrity, signal integrity, and EMI analysis of IC packages and PCB. This tool solves power delivery systems and high-speed channels in electronic devices. It can be accessed from PyAEDT in Windows only. All setups can be implemented through EDB API.
\begin{lstlisting}[language=Python]
from pyaedt.siwave import Siwave
# this call returns the Edb class initialized on 2023 R1
siwave = Siwave(specified_version="2023.1")
siwave.open_project("pyproject.siw")
siwave.export_element_data("mydata.txt")
siwave.close_project()
\end{lstlisting}


\subsection{References from PyAEDT Documentation}
\begin{itemize}
\item \href{https://aedt.docs.pyansys.com/version/stable/EDBAPI/index.html}{EDB API}
\end{itemize}
\end{multicols}
\vspace{-0.15cm}
\noindent\makebox[\linewidth]{\rule{\paperwidth}{4pt}}
\begin{center}
Getting Started With AEDT \includegraphics[height=\fontcharht\font`\S]{slash.png} Ansys Innovation Courses \includegraphics[height=\fontcharht\font`\S]{slash.png} %Visit \code{\href{https://courses.ansys.com/}{ansys.com/courses}}
\end{center}
\end{document}