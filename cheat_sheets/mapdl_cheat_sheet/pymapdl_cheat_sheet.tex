\documentclass[9pt,landscape]{article}
\usepackage{{./../static/style}}
\pdfinfo{
  /Title (PyMAPDL cheat sheet)
  /Creator (TeX)
  /Producer (pdfTeX 1.40.0)
  /Author (Ansys Inc.,)
  /Subject (PyMAPDL)
  /Keywords (PyMAPDL, MAPDL, FEA, cheat sheet, template)}

\begin{document}
\raggedright
\footnotesize
\begin{center}
     \Huge{\textbf{PyMAPDL cheat sheet}} \\
\end{center}
\begin{center}
     \small{\textbf{Version: 0.65.0 (stable)}} \\
\end{center}
\AddToShipoutPicture*
	{\put(670,577.5){\includegraphics[height = 1.2cm]{ansys.png}}}
\AddToShipoutPictureBG*{\includegraphics[width=\paperwidth]{bground.png}}
\vspace{-0.15cm}
\noindent\makebox[\linewidth]{\rule{\paperwidth}{2pt}}

\begin{multicols}{3}
% multicol parameters
% These lengths are set only within the two main columns
%\setlength{\columnseprule}{0.25pt}
\setlength{\premulticols}{1pt}
\setlength{\postmulticols}{1pt}
\setlength{\multicolsep}{1pt}
\setlength{\columnsep}{2pt}

\section{\includegraphics[height=\fontcharht\font`\S]{slash.png} Launch PyMAPDL}
Launch a PyMAPDL instance locally and exit it:
\begin{lstlisting}[language=Python]
# Launch an instance
from ansys.mapdl.core import launch_mapdl
mapdl = launch_mapdl()
# Exit the instance
mapdl.exit()
\end{lstlisting}

Specify a job name, number of processors, and working directory:
\begin{lstlisting}[language=Python]
jname = 'user_jobname'
path = '<path of directory>'
mapdl = launch_mapdl(nproc=2, run_location=path, jobname=jname)
\end{lstlisting}

Connect to an existing instance of MAPDL at IP address 192.168.1.30 and port 50001:
\begin{lstlisting}[language=Python]
mapdl = launch_mapdl(start_instance=False,
    ip='192.168.1.30', 50001)
\end{lstlisting}
Create and exit a pool of instances:
\begin{lstlisting}[language=Python]
# Create a pool of 10 instances
from ansys.mapdl.core import LocalMapdlPool
pool = mapdl.LocalMapdlPool(10)
# Exit the pool
pool.exit()
\end{lstlisting}

\section{\includegraphics[height=\fontcharht\font`\S]{slash.png} PyMAPDL commands}
PyMAPDL commands are Python statements that act as a wrapper for APDL commands. For example, \textt{ESEL, s, type, 1} is translated as:
\begin{lstlisting}[language=Python]
mapdl.esel('s', 'type', vmin=1) 
\end{lstlisting}

Commands that start with * or / have these characters removed:
\begin{lstlisting}[language=Python]
mapdl.prep7()	        # /PREP7
mapdl.get()	        # *GET
\end{lstlisting}

In cases where removing * or / causes conflicts with other commands, a prefix "slash" or "star" is added:
\begin{lstlisting}[language=Python]
mapdl.solu()		# SOLU
mapdl.slashsolu()	# /SOLU

mapdl.vget()		# VGET
mapdl.starvget()	# *VGET
\end{lstlisting} 

\columnbreak
Convert an existing APDL script to PyMAPDL format:
\begin{lstlisting}[language=Python]
inputfile = 'ansys_inputfile.inp'
pyscript = 'pyscript.py'
mapdl.convert_script(inputfile, pyscript)
\end{lstlisting} 

\section{\includegraphics[height=\fontcharht\font`\S]{slash.png} MAPDL class}
Load a table from Python to MAPDL:
\begin{lstlisting}[language=Python]
mapdl.load_table(name, array)
\end{lstlisting} 

Access from or write parameters to the MAPDL database:
\begin{lstlisting}[language=Python]
# Save a parameter to a NumPy array
nparray = mapdl.parameters['displ_load']
# Create a parameter from a NumPy array
mapdl.parameters['exp_disp'] = nparray
\end{lstlisting} 

Access information using \code{*GET} and \code{*VGET} and store it in NumPy arrays:
\begin{lstlisting}[language=Python]
# Run *GET command and return a Python value
mapdl.get_value(entity='NODE', item1='COUNT')

# Run *VGET command and return a Python array
mapdl.get_array(entity='NODE', item1='NLIST')
\end{lstlisting} 

\section{\includegraphics[height=\fontcharht\font`\S]{slash.png} Mesh class}
Store the finite element mesh as a VTK \textt{UnstructuredGrid} data object:
\begin{lstlisting}[language=Python]
grid = mapdl.mesh.grid
\end{lstlisting} 

Save element and node numbers to Python arrays:
\begin{lstlisting}[language=Python]
# Get an array of the nodal coordinates
nodes = mapdl.mesh.nodes

# Save node numbers of selected nodes to an array
node_num = mapdl.mesh.nnum
# Save node numbers of all nodes to an array
node_num_all = mapdl.mesh.nnum_all

# Get element numbers of currently selected elements
elem_num = mapdl.mesh.enum
# Get all element numbers, including those not selected
elem_num_all = mapdl.mesh.enum_all
\end{lstlisting} 
\vfill

\columnbreak

\section{\includegraphics[height=\fontcharht\font`\S]{slash.png} Post-processing class}
%To plot results the general form is: \code{mapdl.postprocessing.result_name}
The \texttt{post_processing] class is used for plotting and saving results to NumPy arrays.
\begin{lstlisting}[language=Python]
from ansys.mapdl.core import post_processing
mapdl.post1()
mapdl.set(1, 2)
# Plot the nodal equivalent stress
mapdl.post_processing.plot_nodal_eqv_stress()
# Save nodal equivalent stresses to a Python array
nod_eqv_stress = post_processing.nodal_eqv_stress()
# Plot contour legend using dictionary
mapdl.allsel()
sbar_kwargs = {"color": "black",
               "title": "Equivalent Stress (psi)",
	       "vertical": False,
	       "n_labels": 6}
post_processing.plot_nodal_eqv_stress(
    cpos='xy', background='white',
    edge_color='black', show_edges=True,
    scalar bar_args=sbar_kwargs, n_colors=9)
\end{lstlisting} 
\vfill

\section{\includegraphics[height=\fontcharht\font`\S]{slash.png} Plotting class}
Use PyVista to interpolate data, saving the resulting stress and storing it in the underlying \textt{UnstructuredGrid} object:
\begin{lstlisting}[language=Python]
pl = pyvista.Plotter()
pl0 = post_processing.plot_nodal_stress(
    return_plotter=True)
pl.add(pl0.mesh)
pl.show()
\end{lstlisting} 

\begin{lstlisting}[language=Python]
# Plot currently selected elements
mapdl.eplot(show_node_numbering, vtk)
# Plot selected volumes
mapdl.vplot(nv1, nv2, ninc, degen, scale, ...)
# Display selected areas
mapdl.aplot(na1, na2, ninc, degen, scale, ...)
# Display selected lines without MAPDL plot symbols
mapdl.lplot(vtk=True, cpos='xy', line_width=10)
# Save PNG file of line plot with MAPDL coordinate symbol
mapdl.psymb('CS', 1)
mapdl.lplot(vtk=False)
\end{lstlisting} 
\vfill

\subsection{References from PyMAPDL documentation}
\begin{itemize}
\item \href{https://mapdl.docs.pyansys.com/version/stable/getting_started/index.html}{\color{blue}{Getting started}}
\item \href{https://mapdl.docs.pyansys.com/version/stable/mapdl_commands/index.html}{\color{blue}{MAPDL commands}}
\item \href{https://mapdl.docs.pyansys.com/version/stable/api/index.html}{\color{blue}{API reference}}
\end{itemize}
\end{multicols}
\vspace{-0.15cm}
\noindent\makebox[\linewidth]{\rule{\paperwidth}{4pt}}
\begin{center}
Getting started with PyMAPDL \includegraphics[height=\fontcharht\font`\S]{slash.png} \href{https://github.com/ansys/pymapdl}{{\color{blue}PyMAPDL on GitHub}} \includegraphics[height=\fontcharht\font`\S]{slash.png} Visit \code{\href{https://mapdl.docs.pyansys.com/}{{\color{blue}mapdl.docs.pyansys.com}}}
\end{center}
\end{document}
