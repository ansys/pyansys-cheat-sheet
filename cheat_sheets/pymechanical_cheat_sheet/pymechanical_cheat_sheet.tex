\documentclass[9pt,landscape]{article}
\usepackage{{./../static/style}}
\pdfinfo{
  /Title (PyMechanical Cheat Sheet)
  /Creator (TeX)
  /Producer (pdfTeX 1.40.0)
  /Author (Ansys)
  /Subject (PyAnsys)
  /Keywords (PyAnsys, Cheat sheet, template, PyMechanical, Mechanical)}


\begin{document}
\raggedright
\footnotesize

% Add the title of cheat sheet here
% ----------------------------------------
\begin{center}
     \Huge{\textbf{PyMechanical Cheat sheet}} \\
\end{center}


\AddToShipoutPicture*
  {\put(670,577.5){\includegraphics[height = 1.2cm]{ansys.png}}}
\AddToShipoutPictureBG*{\includegraphics[width=\paperwidth]{bground.png}}
\vspace{-0.15cm}
\noindent\makebox[\linewidth]{\rule{\paperwidth}{2pt}}

\begin{multicols}{3}
\setlength{\premulticols}{1pt}
\setlength{\postmulticols}{1pt}
\setlength{\multicolsep}{1pt}
\setlength{\columnsep}{2pt}

% session starts here. 
% First colomn
% --------------------------------------------------------------------------------
\vfill
\section{\includegraphics[height=\fontcharht\font`\S]{slash.png} Connecting to Remote Mechanical Session)}
METHOD 1 (local) :
"Launch" Mechanical and "Connect" :

\pythoncode{scripts/generated_scripts/pymechanical_script_0.py}

METHOD 2 (local) :
"Launch" Standalone Mechancial in server mode...

\pythoncode{scripts/generated_scripts/pymechanical_script_1.py}

... and manually "Connect" to it : Mechanical()

\pythoncode{scripts/generated_scripts/pymechanical_script_2.py}

METHOD 3 (remote) :
"Launch" Standalone Mechancial in server mode on a remote machine...

\pythoncode{scripts/generated_scripts/pymechanical_script_3.py}

... and manually "Connect" to it from a local client

\pythoncode{scripts/generated_scripts/pymechanical_script_4.py}


% row 2 col 1
\section{\includegraphics[height=\fontcharht\font`\S]{slash.png} Launch by Version}

Verify Version of Mechanical used:

\pythoncode{scripts/generated_scripts/pymechanical_script_1.py}

To Launch a Specefic Version of Mechanical

\pythoncode{scripts/generated_scripts/pymechanical_script_1.py}


% row 3 col 1
\section{\includegraphics[height=\fontcharht\font`\S]{slash.png} Launch GUI}

To Observe the effect of your script by opening Mechanical UI:
\pythoncode{scripts/generated_scripts/pymechanical_script_1.py}



% Second column
% --------------------------------------------------------------------------------
% row 1 col 2
\vfill


\section{\includegraphics[height=\fontcharht\font`\S]{slash.png} Connecting to Remote Mechanical Session)}
METHOD 1 (local) :
"Launch" Mechanical and "Connect" :

\pythoncode{scripts/generated_scripts/pymechanical_script_0.py}

METHOD 2 (local) :
"Launch" Standalone Mechancial in server mode...

\pythoncode{scripts/generated_scripts/pymechanical_script_1.py}

... and manually "Connect" to it : Mechanical()

\pythoncode{scripts/generated_scripts/pymechanical_script_2.py}

METHOD 3 (remote) :
"Launch" Standalone Mechancial in server mode on a remote machine...

\pythoncode{scripts/generated_scripts/pymechanical_script_3.py}

... and manually "Connect" to it from a local client

\pythoncode{scripts/generated_scripts/pymechanical_script_4.py}

% Third column
% --------------------------------------------------------------------------------
\vfill
% row 1 col 3
\section{\includegraphics[height=\fontcharht\font`\S]{slash.png} Connecting to Remote Mechanical Session)}
METHOD 1 (local) :
"Launch" Mechanical and "Connect" :

\pythoncode{scripts/generated_scripts/pymechanical_script_0.py}

METHOD 2 (local) :
"Launch" Standalone Mechancial in server mode...

\pythoncode{scripts/generated_scripts/pymechanical_script_1.py}

... and manually "Connect" to it : Mechanical()

\pythoncode{scripts/generated_scripts/pymechanical_script_2.py}

METHOD 3 (remote) :
"Launch" Standalone Mechancial in server mode on a remote machine...

\pythoncode{scripts/generated_scripts/pymechanical_script_3.py}

... and manually "Connect" to it from a local client

\pythoncode{scripts/generated_scripts/pymechanical_script_4.py}


% Add subsection
% This section includes useful links to the documentation.
% Examples: installation, API reference, commands, examples.
% Replace 'name of link' with appropriate display text.

\subsection{References from PyMechanical Documentation}
\begin{itemize}
    \item \href{https://mechanical.docs.pyansys.com/version/stable/getting_started/index.html}{\color{blue}{Getting Started}}
    \item \href{https://mechanical.docs.pyansys.com/version/stable/examples/index.html}{\color{blue}{Examples}}
    \item \href{https://mechanical.docs.pyansys.com/version/stable/api/index.html}{\color{blue}{API Reference}}
\end{itemize}
\end{multicols}

% Footer session of the latex with link to documentation and GitHub page
\vspace{-0.15cm}
\noindent\makebox[\linewidth]{\rule{\paperwidth}{4pt}}
\begin{center}
Getting Started with PyMechanical \includegraphics[height=\fontcharht\font`\S]{slash.png} \href{https://github.com/ansys/pymechanical}{\color{blue}{PyMechanical on GitHub}}} \includegraphics[height=\fontcharht\font`\S]{slash.png} Visit \code{\href{https://mechanical.docs.pyansys.com/}}{\color{blue}{mechanical.docs.pyansys.com}}
\end{center}
\end{document}
