\documentclass[9pt,landscape]{article}
\usepackage{{./../static/style}}
\pdfinfo{
  /Title (PyEnSight Cheat Sheet)
  /Creator (TeX)
  /Producer (pdfTeX 1.40.0)
  /Author (Ansys)
  /Subject (PyEnSight)
  /Keywords (PyEnSight, Cheat sheet, template)}

\begin{document}
\raggedright
\footnotesize

% Add the title of cheat sheet here
% ----------------------------------------
\begin{center}
     \Huge{\textbf{PyEnSight Cheat sheet}} \\
\end{center}
\begin{center}
     \small{\textbf{version:0.5.1 (stable)}} \\
\end{center}

\AddToShipoutPicture*
  {\put(670,577.5){\includegraphics[height = 1.2cm]{ansys.png}}}
\AddToShipoutPictureBG*{\includegraphics[width=\paperwidth]{bground.png}}
\vspace{-0.5cm}
\noindent\makebox[\linewidth]{\rule{\paperwidth}{2pt}}

\begin{multicols}{3}
\setlength{\premulticols}{1pt}
\setlength{\postmulticols}{1pt}
\setlength{\multicolsep}{1pt}
\setlength{\columnsep}{2pt}

% session starts here. 
% First colomn
% --------------------------------------------------------------------------------

\section{\includegraphics[height=\fontcharht\font`\S]{slash.png} Launch EnSight locally}

\pythoncode{scripts/generated_scripts/pyensight_script_0.py}

% row 2 col 1
The following example launches EnSight locally with the full GUI available


\pythoncode{scripts/generated_scripts/pyensight_script_1.py}

\section{\includegraphics[height=\fontcharht\font`\S]{slash.png} Launch EnSight from a Container}


\pythoncode{scripts/generated_scripts/pyensight_script_2.py}


% row 3 col 1
\section{\includegraphics[height=\fontcharht\font`\S]{slash.png} Load an example}

\pythoncode{scripts/generated_scripts/pyensight_script_3.py}

\section{\includegraphics[height=\fontcharht\font`\S]{slash.png} Load EnSight Data, Fluent Data, Ansys Mechanical Data}

The "new\_case" option allows to load multiple datasets in the same EnSight session, for a real multi-physics post-processing experience.

\pythoncode{scripts/generated_scripts/pyensight_script_4.py}


% Second column
% --------------------------------------------------------------------------------
% row 1 col 2
\section{\includegraphics[height=\fontcharht\font`\S]{slash.png} Show a picture of the current session and display in a browser}

\pythoncode{scripts/generated_scripts/pyensight_script_5.py}

% row 2 col 2
\section{\includegraphics[height=\fontcharht\font`\S]{slash.png}  Create a deep-pixel, an animation and a webgl renderable for the current session}

Each of the object returned by show() can still be displayed in a web browser with the browser() method
\pythoncode{scripts/generated_scripts/pyensight_script_6.py}

\section{{\includegraphics[height=\fontcharht\font`\S]{slash.png}  Connect to the EnSight Post Processing session}} \\

The full rendering window of EnSight is available with the "remote" renderable.
The "url" property returns the URL you can use to embed the renderable in your application. It is available to all the renderables.
\pythoncode{scripts/generated_scripts/pyensight_script_7.py}

\section{{\includegraphics[height=\fontcharht\font`\S]{slash.png}  Export a picture, a deep pixel picture and an animation}} \\

\pythoncode{scripts/generated_scripts/pyensight_script_8.py}

\section{{\includegraphics[height=\fontcharht\font`\S]{slash.png}  Select the parts by a specific dimension}} \\


\pythoncode{scripts/generated_scripts/pyensight_script_9.py}

\section{{\includegraphics[height=\fontcharht\font`\S]{slash.png}  Set an isometric view}} \\

\pythoncode{scripts/generated_scripts/pyensight_script_10.py}

\section{{\includegraphics[height=\fontcharht\font`\S]{slash.png}  Create a query on a 1D part and plot it}} \\

\pythoncode{scripts/generated_scripts/pyensight_script_11.py}


\section{{\includegraphics[height=\fontcharht\font`\S]{slash.png}  Save and restore a context in memory and on disk}} \\

\pythoncode{scripts/generated_scripts/pyensight_script_12.py}

\section{{\includegraphics[height=\fontcharht\font`\S]{slash.png}  Common graphics operations}} \\

set\_attr() is a method available for a list of EnSight Objects, while setattr() is a method available for the single object. 
The examples show how to color a variable, change its representation, shading, or finally show the mesh lines.

\pythoncode{scripts/generated_scripts/pyensight_script_13.py}
% Third column
% --------------------------------------------------------------------------------

% row 1 col 3
\
% Add subsection
% This section includes useful links to the documentation.
% Examples: installation, API reference, commands, examples.
% Replace 'name of link' with appropriate display text.

\subsection{References from PyEnSight Documentation}
\begin{itemize}
\item \href{https://ensight.docs.pyansys.com/version/stable/getting_started/index.html}{\color{blue}{Getting Started}}  
\item \href{https://ensight.docs.pyansys.com/version/stable/class_documentation.html}{\color{blue}{EnSight API through PyEnSight}}
\item \href{https://ensight.docs.pyansys.com/version/stable/_examples/index.html}{\color{blue}{PyEnSight Examples}}
\end{itemize}
\end{multicols}

% Footer session of the latex with link to documentation and GitHub page
\vspace{-0.75cm}
\noindent\makebox[\linewidth]{\rule{\paperwidth}{4pt}}
\begin{center}
Getting Started with PyEnSight \includegraphics[height=\fontcharht\font`\S]{slash.png} \href{https://github.com/ansys/pyensight}{\color{blue}{PyEnSight on GitHub}}} \includegraphics[height=\fontcharht\font`\S]{slash.png} Visit \code{\href{https://ensight.docs.pyansys.com/}}{\color{blue}{ensight.docs.pyansys.com/}}
\end{center}
\end{document}
