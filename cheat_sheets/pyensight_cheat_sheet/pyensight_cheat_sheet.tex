\documentclass[9pt,landscape]{article}
\usepackage{{./../static/style}}
\pdfinfo{
  /Title (PyEnSight Cheat Sheet)
  /Creator (TeX)
  /Producer (pdfTeX 1.40.0)
  /Author (Ansys)
  /Subject (PyEnSight)
  /Keywords (PyEnSight, Cheat sheet, template)}

\begin{document}
\raggedright
\footnotesize

% Add the title of cheat sheet here
% ----------------------------------------
\begin{center}
     \Huge{\textbf{PyEnSight Cheat sheet}} \\
\end{center}
\begin{center}
     \small{\textbf{version:0.5.1 (stable)}} \\
\end{center}

\AddToShipoutPicture*
  {\put(670,577.5){\includegraphics[height = 1.2cm]{ansys.png}}}
\AddToShipoutPictureBG*{\includegraphics[width=\paperwidth]{bground.png}}
\vspace{-0.3cm}
\noindent\makebox[\linewidth]{\rule{\paperwidth}{2pt}}
\begin{multicols}{3}
\setlength{\premulticols}{0.5pt}
\setlength{\postmulticols}{0.5pt}
\setlength{\multicolsep}{0.5pt}
\setlength{\columnsep}{1pt}

% session starts here. 
% First colomn
% --------------------------------------------------------------------------------

\section{\includegraphics[height=\fontcharht\font`\S]{slash.png} Launching EnSight}


To launch EnSight locally:

\pythoncode{scripts/generated_scripts/pyensight_script_0.py}

% row 2 col 1
To launch EnSight locally using the launcher:
\pythoncode{scripts/generated_scripts/pyensight_script_1.py}

To launch EnSight locally from a container:


\pythoncode{scripts/generated_scripts/pyensight_script_2.py}


% row 3 col 1
\section{\includegraphics[height=\fontcharht\font`\S]{slash.png} Loading data}

To load a design point post-processing example:

\pythoncode{scripts/generated_scripts/pyensight_script_3.py}

To load EnSight, Fluent, or Mechanical data:

\pythoncode{scripts/generated_scripts/pyensight_script_4.py}


% Second column
% --------------------------------------------------------------------------------
% row 1 col 2
\section{\includegraphics[height=\fontcharht\font`\S]{slash.png} PyEnSight Renderables}

To create an image Renderable:


\pythoncode{scripts/generated_scripts/pyensight_script_5.py}

% row 2 col 2
To create a deep-pixel picture, an animation and a webgl Renderable:
\pythoncode{scripts/generated_scripts/pyensight_script_6.py}

To display the EnSight Rendering Window:
\pythoncode{scripts/generated_scripts/pyensight_script_7.py}


\section{{\includegraphics[height=\fontcharht\font`\S]{slash.png}  Working with EnSight}}

To export a picture, a deep-pixel picture and an animation:
\pythoncode{scripts/generated_scripts/pyensight_script_8.py}
To select the parts by a specific dimension:
\pythoncode{scripts/generated_scripts/pyensight_script_9.py}
To set an isometric view: 
\pythoncode{scripts/generated_scripts/pyensight_script_10.py}
To query a 1D part and plot the result:
\pythoncode{scripts/generated_scripts/pyensight_script_11.py}


\section{{\includegraphics[height=\fontcharht\font`\S]{slash.png}  Saving and restoring }} \\

To save in memory:
\pythoncode{scripts/generated_scripts/pyensight_script_12.py}

To save on disk:
\pythoncode{scripts/generated_scripts/pyensight_script_13.py}

To restore from memory:
\pythoncode{scripts/generated_scripts/pyensight_script_14.py}

To restore from disk:
\pythoncode{scripts/generated_scripts/pyensight_script_15.py}

\section{{\includegraphics[height=\fontcharht\font`\S]{slash.png}  Common graphics operations}} 


To color by a variable:
\pythoncode{scripts/generated_scripts/pyensight_script_16.py}

To change the representation:
\pythoncode{scripts/generated_scripts/pyensight_script_17.py}

To show the mesh lines:
\pythoncode{scripts/generated_scripts/pyensight_script_18.py}
% Third column
% --------------------------------------------------------------------------------

% row 1 col 3
\
% Add subsection
% This section includes useful links to the documentation.
% Examples: installation, API reference, commands, examples.
% Replace 'name of link' with appropriate display text.

\subsection{References from PyEnSight Documentation}
\begin{itemize}
\item \href{https://ensight.docs.pyansys.com/version/stable/getting_started/index.html}{\color{blue}{Getting Started}}  
\item \href{https://ensight.docs.pyansys.com/version/stable/class_documentation.html}{\color{blue}{EnSight API through PyEnSight}}
\item \href{https://ensight.docs.pyansys.com/version/stable/_examples/index.html}{\color{blue}{PyEnSight Examples}}
\end{itemize}
\end{multicols}

% Footer session of the latex with link to documentation and GitHub page
\vspace{-0.2cm}
\noindent\makebox[\linewidth]{\rule{\paperwidth}{4pt}}
\begin{center}
Getting Started with PyEnSight \includegraphics[height=\fontcharht\font`\S]{slash.png} \href{https://github.com/ansys/pyensight}{\color{blue}{PyEnSight on GitHub}}} \includegraphics[height=\fontcharht\font`\S]{slash.png} Visit \code{\href{https://ensight.docs.pyansys.com/}}{\color{blue}{ensight.docs.pyansys.com/}}
\end{center}
\end{document}
