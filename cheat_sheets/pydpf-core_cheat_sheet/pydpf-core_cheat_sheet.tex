\documentclass[9pt,landscape]{article}
\usepackage{{./../static/style}}
\pdfinfo{
  /Title (PyDPF-Core Cheat Sheet)
  /Creator (TeX)
  /Producer (pdfTeX 1.40.0)
  /Author (Ansys)
  /Subject (PyDPF)
  /Keywords (PyAnsys, PyDPF, DPF, Cheat sheet, template)}

\begin{document}
\raggedright
\footnotesize

% Add the title of cheat sheet here
% ----------------------------------------

\begin{center}
     \Huge{\textbf{PyDPF-Core Cheat sheet}} \\
\end{center}
\AddToShipoutPicture*
  {\put(670,577.5){\includegraphics[height = 1.2cm]{ansys.png}}}
\AddToShipoutPictureBG*{\includegraphics[width=\paperwidth]{bground.png}}
\vspace{-0.15cm}
\noindent\makebox[\linewidth]{\rule{\paperwidth}{2pt}}

\begin{multicols}{3}
\setlength{\premulticols}{1pt}
\setlength{\postmulticols}{1pt}
\setlength{\multicolsep}{1pt}
\setlength{\columnsep}{2pt}

% session starts here. 
% First colomn
% --------------------------------------------------------------------------------

\section{\includegraphics[height=\fontcharht\font`\S]{slash.png} Importing a Model}
The DPF Model gives an overview of the result files content.
\begin{lstlisting}[language=Python]
from ansys.dpf import core as dpf
model = dpf.Model(r"D:\Path\to\my\file.rst")
print(model)
\end{lstlisting}


\section{\includegraphics[height=\fontcharht\font`\S]{slash.png} Fetching Metadata}
Fetch all the results available in the results file.
\begin{lstlisting}[language=Python]
metadata = model.metadata
print(metadata.result_info)
\end{lstlisting}

Print the time or frequency of the results.
\begin{lstlisting}[language=Python]
print(metadata.time_freq_support)
\end{lstlisting}

Print available named selections.
\begin{lstlisting}[language=Python]
print(metadata.available_named_selections)
\end{lstlisting}


\section{\includegraphics[height=\fontcharht\font`\S]{slash.png} Managing Data Sources}
Manage result files by creating a DPF DataSources object from scratch or from  a model.
\begin{lstlisting}[language=Python]
from ansys.dpf import core as dpf
data_src = dpf.DataSources(r"D:\Path\to\my\file.rst")
# or
data_src = model.metadata.data_sources
\end{lstlisting}


\section{\includegraphics[height=\fontcharht\font`\S]{slash.png} Read Results }
Read displacement results by instantiating a displacement operator and print the fields container.
\begin{lstlisting}[language=Python]
disp_op = model.results.displacement()
disp = disp_op.outputs.fields_container()
print(disp)
\end{lstlisting}

Get field by Time ID = 1.
\begin{lstlisting}[language=Python]
disp_1 = disp.get_field_by_time_id(1).data
print(disp_1)
\end{lstlisting}

% Second column
% --------------------------------------------------------------------------------
\vfill
\columnbreak

\section{\includegraphics[height=\fontcharht\font`\S]{slash.png} Read the Mesh }
Get mesh information, get the list of node IDs, get the list of element IDs.
\begin{lstlisting}[language=Python]
mesh = metadata.meshed_region
print(mesh)
node_IDs = mesh.nodes.scoping.ids
element_IDs = mesh.elements.scoping.ids
\end{lstlisting}

Get node and element using their IDs.
\begin{lstlisting}[language=Python]
# Node ID 1
node = mesh.nodes.node_by_id(1)
# Element ID 1
element = mesh.elements.element_by_id(1)
\end{lstlisting}


\section{\includegraphics[height=\fontcharht\font`\S]{slash.png} Scope Results over Time/Space Domains }
Scoping results over particular time IDs make data fetching faster.
\begin{lstlisting}[language=Python]
time_sets = [1, 3, 10]  # Data for time IDs
disp_op = model.results.displacement()
disp_op.inputs.time_scoping(time_sets)
\end{lstlisting}

Scoping over particular elements.
\begin{lstlisting}[language=Python]
eids = range(262, 272)  # Element scoping
element_scoping = dpf.Scoping(ids=eids)
element_scoping.location = dpf.locations.elemental

s_y = dpf.operators.result.stress_Y()
s_y.inputs.mesh_scoping.connect(element_scoping)
\end{lstlisting}

Retrieve a mesh from a scoping.
\begin{lstlisting}[language=Python]
mesh_from_scoping_op = dpf.operators.mesh.from_scoping(mesh, element_scoping)
cropped_mesh = mesh_from_scoping_op.outputs.mesh()
\end{lstlisting}


\section{\includegraphics[height=\fontcharht\font`\S]{slash.png} Get a data Field }
Apply operators and extract the resulting data Field.
\begin{lstlisting}[language=Python]
norm_op2 = dpf.operators.math.norm_fc()
disp = model.results.displacement()
norm_op2.inputs.connect(disp.outputs)
field_norm_disp = norm_op2.outputs.fields_container()[0]
\end{lstlisting}

% Third column
% --------------------------------------------------------------------------------
\vfill
\columnbreak

\section{\includegraphics[height=\fontcharht\font`\S]{slash.png} Plot Contour Results }
Plot a Field on its mesh, show the minimum and maximum.
\begin{lstlisting}[language=Python]
field_norm_disp.plot(show_max=True, show_min=True,
    title='Field', text='Field plot')
\end{lstlisting}
Take a screenshot from a specific camera position.
\begin{lstlisting}[language=Python]
cpos = [(0.123, 0.095, 1.069), (-0.121, -0.149, 0.825), (0.0, 0.0, 1.0)]
field.plot(off_screen=True, notebook=False, screenshot='field_plot.png', cpos=cpos)
\end{lstlisting}

\section{\includegraphics[height=\fontcharht\font`\S]{slash.png} Workflows }
Combine operators to create Workflows.
\begin{lstlisting}[language=Python]
disp_op = dpf.operators.result.displacement()
max_fc_op = dpf.operators.min_max.min_max_fc(disp_op)
workflow = dpf.Workflow()
workflow.add_operators([disp_op,max_fc_op])
workflow.set_input_name("data_sources", disp_op.inputs.data_sources)
workflow.set_output_name("min", max_fc_op.outputs.field_min)
workflow.set_output_name("max", max_fc_op.outputs.field_max)
\end{lstlisting}

Use the workflow.
\begin{lstlisting}[language=Python]
data_src = dpf.data_sources.DataSources(r"D:\Path\to\my\file.rst")
workflow.connect("data_sources", data_src)
min = workflow.get_output("min", dpf.types.field)
max = workflow.get_output("max", dpf.types.field)
\end{lstlisting}


% Add subsection
% This section includes useful links to the documentation.
% Examples: installation, API reference, commands, examples.
% Replace 'name of link' with appropriate display text.

\subsection{References from PyDPF-Core Documentation}
\begin{itemize}
\item \href{https://dpf.docs.pyansys.com/version/stable/getting_started/index.html}{\color{blue}{Getting Started}}
\item \href{https://dpf.docs.pyansys.com/version/stable/examples/index.html}{\color{blue}{Examples}}
\item \href{https://dpf.docs.pyansys.com/version/stable/api/index.html}{\color{blue}{API Reference}}
\item \href{https://dpf.docs.pyansys.com/version/stable/operator_reference.html}{\color{blue}{Operators Reference}}
\end{itemize}
\end{multicols}

% Footer session of the latex with link to documentation and GitHub page
\vspace{-0.15cm}
\noindent\makebox[\linewidth]{\rule{\paperwidth}{4pt}}
\begin{center}
Getting Started with PyDPF-Core \includegraphics[height=\fontcharht\font`\S]{slash.png} \href{https://github.com/ansys/pydpf-core}{{\color{blue}PyDPF-Core on GitHub}} \includegraphics[height=\fontcharht\font`\S]{slash.png} Visit \code{\href{https://dpf.docs.pyansys.com/}}{{\color{blue}dpf.docs.pyansys.com}}
\end{center}
\end{document}