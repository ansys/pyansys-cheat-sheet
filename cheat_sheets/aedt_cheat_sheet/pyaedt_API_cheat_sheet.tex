\documentclass[9pt,landscape]{article}
\usepackage{{./../static/style}}

\pdfinfo{
  /Title (PyAnsys cheat sheet)
  /Creator (TeX)
  /Producer (pdfTeX 1.40.0)
  /Author (Ansys Inc.,)
  /Subject (PyAEDT)
  /Keywords (PyAEDT, AEDT, cheat sheet, template)}

% -----------------------------------------------------------------------

\begin{document}
%\raggedright
\footnotesize
\justifying
\begin{center}
     \Huge{\textbf{PyAEDT-API cheat sheet}} \\
\end{center}
\begin{center}
\small{\textbf{Version: 0.6.78 (stable)}} \\
\end{center}
\AddToShipoutPicture*
{\put(670,577.5){\includegraphics[height = 1.2cm]{ansys.png}}}
\AddToShipoutPictureBG*{\includegraphics[width=\paperwidth]{bground.png}}
\vspace{-0.15cm}
\noindent\makebox[\linewidth]{\rule{\paperwidth}{2pt}}

\begin{multicols}{3}
% multicol parameters
% These lengths are set only within the two main columns
%\setlength{\columnseprule}{0.25pt}
\setlength{\premulticols}{1pt}
\setlength{\postmulticols}{1pt}
\setlength{\multicolsep}{1pt}
\setlength{\columnsep}{2pt}

\section{\includegraphics[height=\fontcharht\font`\S]{slash.png} Launch PyAEDT}
Launch an HFSS instance locally:
\begin{lstlisting}[language=Python]
import pyaedt
hfss = pyaedt.Hfss(specified_version="2023.1",
	non_graphical=False,
	new_desktop_session=True,
	projectname="Project_name",
	designname="Design_name")
\end{lstlisting}
Exit your local instance:
\begin{lstlisting}[language=Python]
hfss.release_desktop()
\end{lstlisting}
%Optionally, AEDT can be launched at the client server. To connect an existing instance of AEDT at  \textbf{"server\_name"} and port \textbf{"50001"}
%\begin{lstlisting}[language=Python]
%	cl1 = create_session("server_name", launch_aedt_on_server=True,
%	aedt_port=50001, non_graphical=True)
%\end{lstlisting}
\section{\includegraphics[height=\fontcharht\font`\S]{slash.png} Variable class}
The \texttt{hfss.variable$\_$manager} class handles all variables.
\newline
Create a variable that only applies to this design:
\begin{lstlisting}[language=Python]
hfss["dim"] = "1mm"
\end{lstlisting}
Create a variable that applies at a project level:
\begin{lstlisting}[language=Python]
hfss["$dim"] = "1mm"
\end{lstlisting}
\section{\includegraphics[height=\fontcharht\font`\S]{slash.png} Material class}
The \texttt{hfss.materials} class is used to access the materials library.
\newline
Add a new material: 
\begin{lstlisting}[language=Python]
my_mat = hfss.materials.add_material("myMat")
my_mat.permittivity = 3.5
my_mat.conductivity = 450000
my_mat.permeability = 1.5
\end{lstlisting}
%A Python dictionary of material properties can be added while adding the material itself.
%\begin{lstlisting}[language=Python]
%dic = {"permittivity":4.5, "permeability":1.5,  "conductivity":45000}
%my_mat = hfss.materials.add_material("myMat2", props = dic)
%\end{lstlisting}

\section{\includegraphics[height=\fontcharht\font`\S]{slash.png} Geometry creation}
The \texttt{hfss.modeler} class contains all methods and properties needed to edit a modeler, including those for primitives.
\newline
\\
Draw a box at (x\_pos, y\_pos, z\_pos) position with (x\_dim, y\_dim, z\_dim) dimensions:
\begin{lstlisting}[language=Python]
box = hfss.modeler.create_box([x_pos,y_pos,z_pos], [x_dim,y_dim,z_dim],name="airbox", matname="air")
\end{lstlisting}
Create a spiral geometry made of copper:
\begin{lstlisting}[language=Python]
ind = hfss.modeler.create_spiral(
	internal_radius=rin, width=width,
	spacing=spacing, turns=Nr,
	faces=Np, thickness=thickness,
	material="copper",name="Inductor1")
\end{lstlisting}
\columnbreak
The \texttt{Object3d objects}, \texttt{box} and \texttt{ind}, contain a lot of methods and properties related to that object, including faces, vertices, colors, and materials.

\section{\includegraphics[height=\fontcharht\font`\S]{slash.png} Boundary creation}
Create an open region:
\begin{lstlisting}[language=Python]
hfss.create_open_region(Frequency="1GHz")
\end{lstlisting}
Assign a radiation boundary:
\begin{lstlisting}[language=Python]
hfss.assign_radiation_boundary_to_objects("airbox")
\end{lstlisting}

\section{\includegraphics[height=\fontcharht\font`\S]{slash.png} Port definition}
Common port types in HFSS are lumped port and wave port.
\newline
\\
Define a lumped port:
\begin{lstlisting}[language=Python]
box1 = hfss.modeler.create_box([0,0,50], [10,10,5], "Box1","copper")
box2 = hfss.modeler.create_box([0,0,60], [10,10,5], "Box2","copper")
port_L = hfss.lumped_port(signal="Box1", reference="Box2", integration_line=hfss.AxisDir.XNeg, impedance=50, name="LumpedPort", renormalize=True, deembed=False)
\end{lstlisting}
Define a wave port:
\begin{lstlisting}[language=Python]
port_W = hfss.wave_port("Box1", "Box2", name="WavePort", integration_line=1)
\end{lstlisting}

\section{\includegraphics[height=\fontcharht\font`\S]{slash.png} Setup class}
The \texttt{hfss.create.setup} class is used to define the solution setup:

\begin{lstlisting}[language=Python]
setup = hfss.create_setup("MySetup")
setup.props["Frequency"] = "50MHz"
setup["MaximumPasses"] = 10
hfss.create_linear_count_sweep(setupname="any", unit="MHz", freqstart=0.1, freqstop=100, num_of_freq_points=100, sweepname="sweep1", sweep_type="Interpolating", save_fields=False)
\end{lstlisting}
Access the parametric sweep:
\begin{lstlisting}[language=Python]
hfss.parametrics
\end{lstlisting}
Access the optimizations:
\begin{lstlisting}[language=Python]
hfss.optimizations
\end{lstlisting}
\columnbreak
\section{\includegraphics[height=\fontcharht\font`\S]{slash.png} Mesh class}
The \texttt{hfss.mesh} module manages the mesh functions:
\begin{lstlisting}[language=Python]
hfss.mesh.assign_initial_mesh_from_slider(level=6) # Set the slider level to 6
# Assign model resolutions
hfss.mesh.assign_model_resolution(names=[object1.name, object2.name], defeature_length=None)
# Assign mesh length to \texttt{object1} faces
hfss.mesh.assign_length_mesh(names=object1.faces, isinside=False, maxlength=1, maxel=2000)
\end{lstlisting}

\section{\includegraphics[height=\fontcharht\font`\S]{slash.png} Analyze class}
The \texttt{analyze] class is used to analyze a solution setup (\texttt{mysetup}) in an HFSS design:
\begin{lstlisting}[language=Python]
hfss.analyze_setup("mysetup")
\end{lstlisting}
%where, 'mysetup' is the solution setup
\section{\includegraphics[height=\fontcharht\font`\S]{slash.png} Post class}
The \texttt{post} class has methods for creating and editing plots in AEDT:
\begin{lstlisting}[language=Python]
plotf = hfss.post.create_fieldplot_volume(object_list, quantityname, setup_name, intrinsic_dict) # This call returns a FieldPlot object
my_data = hfss.post.get_solution_data(expression=trace_names) # This call returns a SolutionData object
standard_report = hfss.post.report_by_category.standard("db(S(1,1))") # This call returns a new standard report object
standard_report.create() # This call creates a report
solution_data = standard_report.get_solution_data()
\end{lstlisting}

\section{\includegraphics[height=\fontcharht\font`\S]{slash.png} Call AEDT-API with PyAEDT}
Most core functionality can be called directly through PyAEDT, but additional features can be added by converting the corresponding AEDT-API methods.
\newline
\\
For example, access the \texttt{Optimetrics} module:
\begin{lstlisting}[language=Python]
omodule = hfss.odesign.GetModule("Optimetrics")
\end{lstlisting}

\subsection{References from PyAEDT documentation}
\begin{itemize}
\item \href{https://aedt.docs.pyansys.com/version/stable/Getting_started/index.html}{\color{blue}{Getting started}}
\item \href{https://aedt.docs.pyansys.com/version/stable/User_guide/index.html}{\color{blue}{User guide}}
\item \href{https://aedt.docs.pyansys.com/version/stable/API/index.html}{\color{blue}{API reference}}
\end{itemize}
\end{multicols}
\vspace{-0.15cm}
\noindent\makebox[\linewidth]{\rule{\paperwidth}{4pt}}
\begin{center}
Getting started with AEDT \includegraphics[height=\fontcharht\font`\S]{slash.png} \href{https://github.com/ansys/pyaedt}{{\color{blue}PyAEDT on GitHub}} \includegraphics[height=\fontcharht\font`\S]{slash.png} Ansys Innovation Courses %Visit \code{\href{https://courses.ansys.com/}{ansys.com/courses}}
\end{center}
\end{document}
