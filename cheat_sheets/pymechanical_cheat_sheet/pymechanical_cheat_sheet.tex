\documentclass[9pt,landscape]{article}
\usepackage{{./../static/style}}
\pdfinfo{
  /Title (PyMechanical cheat sheet)
  /Creator (TeX)
  /Producer (pdfTeX 1.40.0)
  /Author (Ansys)
  /Subject (PyAnsys)
  /Keywords (PyAnsys, PyMechanical, Mechanical, cheat sheet, template)}


\begin{document}
\raggedright
\footnotesize

% Add the title of cheat sheet here
% ----------------------------------------
\begin{center}
     \Huge{\textbf{PyMechanical cheat sheet}} \\
\end{center}

\begin{center}
  \small{\textbf{Version: 0.9 (stable)}} \\
\end{center}

\AddToShipoutPicture*
  {\put(670,577.5){\includegraphics[height = 1.2cm]{ansys.png}}}
\AddToShipoutPictureBG*{\includegraphics[width=\paperwidth]{bground.png}}
\vspace{-0.15cm}
\noindent\makebox[\linewidth]{\rule{\paperwidth}{2pt}}

\begin{multicols}{3}
\setlength{\premulticols}{1pt}
\setlength{\postmulticols}{1pt}
\setlength{\multicolsep}{1pt}
\setlength{\columnsep}{2pt}

% session starts here. 
% First colomn
% --------------------------------------------------------------------------------
\vfill
\section{\includegraphics[height=\fontcharht\font`\S]{slash.png} # A. Connect to a \underline{remote session} of Mechanical from Python}

% \vspace{5mm} %5mm vertical space
% {\color{orange} \rule{\linewidth}{0.1mm}}

\section{\includegraphics[height=\fontcharht\font`\S]{slash.png} Launch and connect to a session}

Launch and connect to Mechanical locally:

\pythoncode{scripts/generated_scripts/pymechanical_script_0.py}

Launch Mechanical from a local or remote terminal:

\pythoncode{scripts/generated_scripts/pymechanical_script_1.py}

Manually connect to this Mechanical session from a local client:

\pythoncode{scripts/generated_scripts/pymechanical_script_2.py}



% row 2 col 1
\section{\includegraphics[height=\fontcharht\font`\S]{slash.png} Launch by version}

Verify the license and version of Mechanical that is used:

\pythoncode{scripts/generated_scripts/pymechanical_script_3.py}

Launch a specific version of Mechanical:

\pythoncode{scripts/generated_scripts/pymechanical_script_4.py}





\section{\includegraphics[height=\fontcharht\font`\S]{slash.png} Launch the  Mechanical UI}

Launch the  Mechanical UI:
\pythoncode{scripts/generated_scripts/pymechanical_script_5.py}

% Second column
% --------------------------------------------------------------------------------
% row 1 col 2

\vfill

\section{\includegraphics[height=\fontcharht\font`\S]{slash.png} Send commands to Mechanical}
Run a single command:

\pythoncode{scripts/generated_scripts/pymechanical_script_6.py}

Execute a block of commands:

\pythoncode{scripts/generated_scripts/pymechanical_script_7.py}

Execute a Python script file:

\pythoncode{scripts/generated_scripts/pymechanical_script_8.py}

Import a Mechancial file and print the count of bodies:

\pythoncode{scripts/generated_scripts/pymechanical_script_9.py}



% \section{\includegraphics[height=\fontcharht\font`\S]{slash.png}Project-specific operations}
Perform project-specific operations:

\pythoncode{scripts/generated_scripts/pymechanical_script_10.py}


% Third column
% --------------------------------------------------------------------------------
% \vfill
% row 1 col 3

\section{\includegraphics[height=\fontcharht\font`\S]{slash.png} # B. Load an  \underline{embedded instance} of Mechanical in Python}
% \vspace{5mm} %5mm vertical space
% {\color{orange} \rule{\linewidth}{0.1mm}}
Embed a Mechanical instance: 

\pythoncode{scripts/generated_scripts/pymechanical_script_11.py}

% To extract the global API entry points, that are available from built-in Mechanical scripting, and merge them into your global Python global variables:
Extract and merge global API entry points:

\pythoncode{scripts/generated_scripts/pymechanical_script_12.py}

Access entry points from Python: 
 

\pythoncode{scripts/generated_scripts/pymechanical_script_13.py}

Import a file and print the count of bodies:

\pythoncode{scripts/generated_scripts/pymechanical_script_14.py}

Turn on warning logging:
\pythoncode{scripts/generated_scripts/pymechanical_script_15.py}

% Add subsection
% This section includes useful links to the documentation.
% Examples: installation, API reference, commands, examples.
% Replace 'name of link' with appropriate display text.

\subsection{References from PyMechanical and Mechanical documentation}
\begin{multicols}{2}
\begin{itemize}
    \item \href{https://mechanical.docs.pyansys.com/version/stable/getting_started/index.html}{\color{blue}{Getting started}}
    \item \href{https://mechanical.docs.pyansys.com/version/stable/examples/index.html}{\color{blue}{Examples}}
    \item \href{https://mechanical.docs.pyansys.com/version/stable/api/index.html}{\color{blue}{API reference}}
    \item \href{https://ansyshelp.ansys.com/account/secured?returnurl=/Views/Secured/corp/v231/en/act_script/act_script.html}{\color{blue}{Scripting in Mechanical Guide}}
\end{itemize}
\end{multicols}
\end{multicols}

% Footer session of the latex with link to documentation and GitHub page
\vspace{-0.15cm}
\noindent\makebox[\linewidth]{\rule{\paperwidth}{4pt}}
\begin{center}
Getting started with PyMechanical \includegraphics[height=\fontcharht\font`\S]{slash.png} \href{https://github.com/ansys/pymechanical}{\color{blue}{PyMechanical on GitHub}}}
\includegraphics[height=\fontcharht\font`\S]{slash.png} Visit \code{\href{https://mechanical.docs.pyansys.com/}}{\color{blue}{mechanical.docs.pyansys.com}}} 
\end{center}
\end{document}
