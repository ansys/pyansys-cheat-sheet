\documentclass[9pt,landscape]{article}
\usepackage{{./../static/style}}
\pdfinfo{
  /Title (PyAnsys Cheat Sheet)
  /Creator (TeX)
  /Producer (pdfTeX 1.40.0)
  /Author (Ansys)
  /Subject (PyAnsys)
  /Keywords (PyAnsys, Cheat sheet, template)}

\begin{document}
\raggedright
\footnotesize

% Add the title of cheat sheet here
% ----------------------------------------

\begin{center}
     \Huge{\textbf{PyAnsys Cheat sheet}} \\
\end{center}
\AddToShipoutPicture*
  {\put(670,577.5){\includegraphics[height = 1.2cm]{ansys.png}}}
\AddToShipoutPictureBG*{\includegraphics[width=\paperwidth]{bground.png}}
\vspace{-0.15cm}
\noindent\makebox[\linewidth]{\rule{\paperwidth}{2pt}}

\begin{multicols}{3}
\setlength{\premulticols}{1pt}
\setlength{\postmulticols}{1pt}
\setlength{\multicolsep}{1pt}
\setlength{\columnsep}{2pt}

% session starts here. 
% First colomn
% --------------------------------------------------------------------------------

\section{\includegraphics[height=\fontcharht\font`\S]{slash.png} Add heading here}
Add description of code block here
\begin{lstlisting}[language=Python]

Add code here

\end{lstlisting}

% Second column
% --------------------------------------------------------------------------------

\vfill
\section{\includegraphics[height=\fontcharht\font`\S]{slash.png}  Add heading here}
Add description of code block here
\begin{lstlisting}[language=Python]

Add code here

\end{lstlisting} 

% Third column
% --------------------------------------------------------------------------------
\vfill
\section{\includegraphics[height=\fontcharht\font`\S]{slash.png}  Add heading here}
Add description of code block here
\begin{lstlisting}[language=Python]

Add code here

\end{lstlisting}

% Add subsection
% This section includes useful links to the documentation.
% Examples: installation, API reference, commands, examples.
% Replace 'name of link' with appropriate display text.

\subsection{References from PyAnsys Documentation}
\begin{itemize}
\item \href{usefullinks}{\color{blue}{name of link}}
\item \href{useful_links}{\color{blue}{name of link}}
\item \href{useful_links}{\color{blue}{name of link}}
\end{itemize}
\end{multicols}

% Footer session of the latex with link to documentation and GitHub page
\vspace{-0.15cm}
\noindent\makebox[\linewidth]{\rule{\paperwidth}{4pt}}
\begin{center}
Getting Started with PyAnsys \includegraphics[height=\fontcharht\font`\S]{slash.png} \href{https://github.com/pyansys}{\color{blue}{PyAnsys on GitHub}} \includegraphics[height=\fontcharht\font`\S]{slash.png} Visit \code{\href{https://dev.docs.pyansys.com/}}{\color{blue}{dev.docs.pyansys.com}}
\end{center}
\end{document}