\documentclass[9pt,landscape]{article}
\usepackage{{./../static/style}}
\pdfinfo{
  /Title (PyPrimeMesh cheat sheet)
  /Creator (TeX)
  /Producer (pdfTeX 1.40.0)
  /Author (Ansys)
  /Subject (PyAnsys)
  /Keywords (PyAnsys, PyPrimeMesh, cheat sheet, template)}

\begin{document}
\raggedright
\footnotesize

% PyPrimeMesh cheat sheet
% ----------------------------------------

\begin{center}
     \Huge{\textbf{PyPrimeMesh cheat sheet}} \\
\end{center}
\begin{center}
  \small{\textbf{Version: 0.3.0 (stable)}} \\
\end{center}
\AddToShipoutPicture*
  {\put(670,577.5){\includegraphics[height = 1.2cm]{ansys.png}}}
\AddToShipoutPictureBG*{\includegraphics[width=\paperwidth]{bground.png}}
\vspace{-0.15cm}
\noindent\makebox[\linewidth]{\rule{\paperwidth}{2pt}}

\begin{multicols}{3}
\setlength{\premulticols}{1pt}
\setlength{\postmulticols}{1pt}
\setlength{\multicolsep}{1pt}
\setlength{\columnsep}{2pt}

% session starts here. 
% First colomn
% --------------------------------------------------------------------------------
% row 1 col 1
\vfill
\section{\includegraphics[height=\fontcharht\font`\S]{slash.png} Launch PyPrimeMesh client}
Launch and exit PyPrimeMesh server from Python in gRPC mode:\\
\begin{lstlisting}[language=Python]
# Launch PyPrimeMesh server 
from ansys.meshing import prime

with prime.launch_prime(timeout=20) as prime_client:
    model = prime_client.model
    # Define script here
    prime_client.exit()
\end{lstlisting}

Launch an instance of PyPrimeMesh at IP address
127.0.0.1 and port 50055 with the number of processes set to 4:
\begin{lstlisting}[language=Python]
with prime.launch_prime(
  ip="127.0.0.1", port=50055, n_procs=4
  ) as prime_client:
  model = prime_client.model
\end{lstlisting}

% row 2 col 1
\section{\includegraphics[height=\fontcharht\font`\S]{slash.png} Read and write files}
Read or write files of different formats based on file extensions:\\
\begin{lstlisting}[language=Python]
from ansys.meshing.prime import lucid

# Define \texttt{mesher} object
mesher = lucid.Mesh(model)
# Read mesh (*.msh) file
mesh_file_name = r"sample1_mesh.msh"
mesher.read(mesh_file_name, append=False)
# Write mesh (*.cdb) file
cdb_file_name = r"sample3_case.cdb"
mesher.write(cdb_file_name)
\end{lstlisting}

\section{\includegraphics[height=\fontcharht\font`\S]{slash.png} Part summary}
Query for the part summary:
\begin{lstlisting}[language=Python]
part = model.get_part_by_name("sample_part")
summary = part.get_summary(prime.PartSummaryParams(model))
print("Total number of cells: ", summary.n_cells)
\end{lstlisting}

\section{\includegraphics[height=\fontcharht\font`\S]{slash.png} Define size controls}
Set global sizing parameters: 

\begin{lstlisting}[language=Python]
model.set_global_sizing_params(
    prime.GlobalSizingParams(min=0.5, max=16.0, growth_rate=1.2)
)
\end{lstlisting}

Define the curvature size control:
\begin{lstlisting}[language=Python]
curvature_control = model.control_data.create_size_control(
  prime.SizingType.CURVATURE
)
control_name = "Curvature_Size_Control"
curvature_control.set_suggested_name(control_name)
scope = prime.ScopeDefinition(
    model,
    evaluation_type=prime.ScopeEvaluationType.LABELS,
    label_expression="*",
)
curvature_control.set_scope(scope)
curvature_control.set_curvature_sizing_params(
    prime.CurvatureSizingParams(model, normal_angle=18)
)
\end{lstlisting}
\section{\includegraphics[height=\fontcharht\font`\S]{slash.png} Generate wrapper surface mesh}
Generate the wrapper surface mesh:
\begin{lstlisting}[language=Python]
mesher.wrap(
  min_size=0.5,
  max_size=16,
  input_parts="flange,pipe",
  use_existing_features=True,
  recompute_remesh_sizes=True,
  remesh_size_controls=[curvature_control],
)
\end{lstlisting}

\section{\includegraphics[height=\fontcharht\font`\S]{slash.png} Generate surface mesh}
Generate the surface mesh based on specified minimum and maximum sizes: 
\begin{lstlisting}[language=Python]
mesher.surface_mesh(
    min_size=0.5,
    max_size=16,
    generate_quads=True,
)
\end{lstlisting}

Generate the surface mesh based on size controls:  
\begin{lstlisting}[language=Python]
control_name = "Curvature_Size_Control"
mesher.surface_mesh_with_size_controls(
    control_name,
)
\end{lstlisting}

\section{\includegraphics[height=\fontcharht\font`\S]{slash.png} Analyze surface mesh}
Generate surface mesh diagnostics: 
\begin{lstlisting}[language=Python]
surface_scope = prime.ScopeDefinition(model, part_expression="*")
diag = prime.SurfaceSearch(model)
diag_params = prime.SurfaceDiagnosticSummaryParams(
    model,
    scope=surface_scope,
    compute_free_edges=True,
)
diag_res = diag.get_surface_diagnostic_summary(diag_params)
print("Number of free faces : ", diag_res.n_free_edges)
\end{lstlisting}

Generate surface mesh quality metrics:
\begin{lstlisting}[language=Python]
face_quality_measures = prime.FaceQualityMeasure.SKEWNESS
quality = prime.SurfaceSearch(model)
quality_params = prime.SurfaceQualitySummaryParams(
    model=model,
    scope=surface_scope,
    face_quality_measures=[face_quality_measures],
    quality_limit=[0.9],
)
qual_summary_res = quality.get_surface_quality_summary(quality_params)
print(
    "Maximum surface skewness : ",
    qual_summary_res.quality_results[0].max_quality,
)
\end{lstlisting}

\section{\includegraphics[height=\fontcharht\font`\S]{slash.png} Generate volume mesh}
Generate a volume mesh:
\begin{lstlisting}[language=Python]
mesher.volume_mesh(
    volume_fill_type=prime.VolumeFillType.HEXCOREPOLY,
)
\end{lstlisting}

% Add subsection
% This section includes useful links to the documentation.
% Examples: installation, API reference, commands, examples.
% Replace 'name of link' with appropriate display text.

\subsection{References from PyAnsys documentation}
\begin{itemize}
\item \href{https://prime.docs.pyansys.com/version/stable/getting_started/index.html}{\color{blue}{Getting started}}
\item \href{https://prime.docs.pyansys.com/version/stable/user_guide/index.html}{\color{blue}{User guide}}
\item \href{https://prime.docs.pyansys.com/version/stable/examples/index.html}{\color{blue}{Examples}}
\end{itemize}
\end{multicols}

% Footer session of the latex with link to documentation and GitHub page
\vspace{-0.15cm}
\noindent\makebox[\linewidth]{\rule{\paperwidth}{4pt}}
\begin{center}
Getting started with PyPrimeMesh \includegraphics[height=\fontcharht\font`\S]{slash.png} \href{https://github.com/ansys/pyprimemesh}{\color{blue}{PyPrimeMesh on GitHub}} \includegraphics[height=\fontcharht\font`\S]{slash.png} Visit \code{\href{https://prime.docs.pyansys.com/}}{\color{blue}{prime.docs.pyansys.com}}
\end{center}
\end{document}
