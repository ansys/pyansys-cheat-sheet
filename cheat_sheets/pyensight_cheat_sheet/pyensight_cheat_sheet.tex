\documentclass[9pt,landscape]{article}
\usepackage{{./../static/style}}
\pdfinfo{
  /Title (PyEnSight cheat sheet)
  /Creator (TeX)
  /Producer (pdfTeX 1.40.0)
  /Author (Ansys)
  /Subject (PyEnSight)
  /Keywords (PyEnSight, cheat sheet, template)}

\begin{document}
\raggedright
\footnotesize

% Add the title of cheat sheet here
% ----------------------------------------
\begin{center}
     \Huge{\textbf{PyEnSight cheat sheet}} \\
\end{center}
\begin{center}
     \small{\textbf{Version: 0.5.1 (stable)}} \\
\end{center}

\AddToShipoutPicture*
  {\put(670,577.5){\includegraphics[height = 1.2cm]{ansys.png}}}
\AddToShipoutPictureBG*{\includegraphics[width=\paperwidth]{bground.png}}
\vspace{-0.3cm}
\noindent\makebox[\linewidth]{\rule{\paperwidth}{2pt}}
\begin{multicols}{3}
\setlength{\premulticols}{0.5pt}
\setlength{\postmulticols}{0.5pt}
\setlength{\multicolsep}{0.5pt}
\setlength{\columnsep}{1pt}

% session starts here. 
% First colomn
% --------------------------------------------------------------------------------

\section{\includegraphics[height=\fontcharht\font`\S]{slash.png} Launch EnSight}


Launch EnSight locally:

\pythoncode{scripts/generated_scripts/pyensight_script_0.py}

% row 2 col 1
Launch EnSight locally using the launcher:
\pythoncode{scripts/generated_scripts/pyensight_script_1.py}

Launch EnSight locally from a container:


\pythoncode{scripts/generated_scripts/pyensight_script_2.py}


% row 3 col 1
\section{\includegraphics[height=\fontcharht\font`\S]{slash.png} Load data}

Load an example for postprocessing a design point:

\pythoncode{scripts/generated_scripts/pyensight_script_3.py}

Load EnSight, Fluent, or Mechanical data:

\pythoncode{scripts/generated_scripts/pyensight_script_4.py}


% Second column
% --------------------------------------------------------------------------------
% row 1 col 2
\section{\includegraphics[height=\fontcharht\font`\S]{slash.png} PyEnSight renderables}

Create an image renderable:


\pythoncode{scripts/generated_scripts/pyensight_script_5.py}

% row 2 col 2
Create a deep-pixel picture, an animation, and a WebGL renderable:
\pythoncode{scripts/generated_scripts/pyensight_script_6.py}

Display the EnSight rendering window:
\pythoncode{scripts/generated_scripts/pyensight_script_7.py}


\section{{\includegraphics[height=\fontcharht\font`\S]{slash.png}  Working with EnSight}}

Export a picture, a deep-pixel picture, and an animation:
\pythoncode{scripts/generated_scripts/pyensight_script_8.py}
Select parts by a specific dimension:
\pythoncode{scripts/generated_scripts/pyensight_script_9.py}
Set an isometric view: 
\pythoncode{scripts/generated_scripts/pyensight_script_10.py}
Query a 1D part and plot the result:
\pythoncode{scripts/generated_scripts/pyensight_script_11.py}


\section{{\includegraphics[height=\fontcharht\font`\S]{slash.png}  Saving and restoring }} \\

Save in memory:
\pythoncode{scripts/generated_scripts/pyensight_script_12.py}

Save on disk:
\pythoncode{scripts/generated_scripts/pyensight_script_13.py}

Restore from memory:
\pythoncode{scripts/generated_scripts/pyensight_script_14.py}

Restore from disk:
\pythoncode{scripts/generated_scripts/pyensight_script_15.py}

\section{{\includegraphics[height=\fontcharht\font`\S]{slash.png}  Common graphics operations}} 


Color by a variable:
\pythoncode{scripts/generated_scripts/pyensight_script_16.py}

Change the representation:
\pythoncode{scripts/generated_scripts/pyensight_script_17.py}

Show the mesh lines:
\pythoncode{scripts/generated_scripts/pyensight_script_18.py}
% Third column
% --------------------------------------------------------------------------------

% row 1 col 3
\
% Add subsection
% This section includes useful links to the documentation.
% Examples: installation, API reference, commands, examples.
% Replace 'name of link' with appropriate display text.

\subsection{References from PyEnSight documentation}
\begin{itemize}
\item \href{https://ensight.docs.pyansys.com/version/stable/getting_started/index.html}{\color{blue}{Getting started}}  
\item \href{https://ensight.docs.pyansys.com/version/stable/class_documentation.html}{\color{blue}{EnSight API through PyEnSight}}
\item \href{https://ensight.docs.pyansys.com/version/stable/_examples/index.html}{\color{blue}{Examples}}
\end{itemize}
\end{multicols}

% Footer session of the latex with link to documentation and GitHub page
\vspace{-0.2cm}
\noindent\makebox[\linewidth]{\rule{\paperwidth}{4pt}}
\begin{center}
Getting started with PyEnSight \includegraphics[height=\fontcharht\font`\S]{slash.png} \href{https://github.com/ansys/pyensight}{\color{blue}{PyEnSight on GitHub}}} \includegraphics[height=\fontcharht\font`\S]{slash.png} Visit \code{\href{https://ensight.docs.pyansys.com/}}{\color{blue}{ensight.docs.pyansys.com/}}
\end{center}
\end{document}
