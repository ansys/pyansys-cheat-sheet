\documentclass[9pt,landscape]{article}
\usepackage{{./../static/style}}
\pdfinfo{
  /Title (PyMechanical Cheat Sheet)
  /Creator (TeX)
  /Producer (pdfTeX 1.40.0)
  /Author (Ansys)
  /Subject (PyAnsys)
  /Keywords (PyAnsys, Cheat sheet, template, PyMechanical, Mechanical)}


\begin{document}
\raggedright
\footnotesize

% Add the title of cheat sheet here
% ----------------------------------------
\begin{center}
     \Huge{\textbf{PyMechanical Cheat sheet}} \\
\end{center}


\AddToShipoutPicture*
  {\put(670,577.5){\includegraphics[height = 1.2cm]{ansys.png}}}
\AddToShipoutPictureBG*{\includegraphics[width=\paperwidth]{bground.png}}
\vspace{-0.15cm}
\noindent\makebox[\linewidth]{\rule{\paperwidth}{2pt}}

\begin{multicols}{3}
\setlength{\premulticols}{1pt}
\setlength{\postmulticols}{1pt}
\setlength{\multicolsep}{1pt}
\setlength{\columnsep}{2pt}

% session starts here. 
% First colomn
% --------------------------------------------------------------------------------
\vfill
\section{\includegraphics[height=\fontcharht\font`\S]{slash.png} # A. PyMechanical as a Client to a \underline{Remote} Mechanical Session}

% \vspace{5mm} %5mm vertical space
% {\color{orange} \rule{\linewidth}{0.1mm}}

\section{\includegraphics[height=\fontcharht\font`\S]{slash.png} Launch and Connect to a Session}

(a). Local : "Launch"  and "Connect" to Mechanical

\pythoncode{scripts/generated_scripts/pymechanical_script_0.py}

(b). Local/Remote : "Launch" Mechanical .....

\pythoncode{scripts/generated_scripts/pymechanical_script_1.py}

... and then  manually "Connect" to it from a local client

\pythoncode{scripts/generated_scripts/pymechanical_script_2.py}



% row 2 col 1
\section{\includegraphics[height=\fontcharht\font`\S]{slash.png} Launch by Version}

Verify Version of Mechanical used:

\pythoncode{scripts/generated_scripts/pymechanical_script_3.py}

Launch a Specific Version of Mechanical

\pythoncode{scripts/generated_scripts/pymechanical_script_4.py}



% Second column
% --------------------------------------------------------------------------------
% row 1 col 2
% \vfill

\section{\includegraphics[height=\fontcharht\font`\S]{slash.png} Launch  Mechanical UI}

Use the batch argument to open Mechanical UI:
\pythoncode{scripts/generated_scripts/pymechanical_script_5.py}

\section{\includegraphics[height=\fontcharht\font`\S]{slash.png} Send Commands to Mechanical}
Execute a Single command :

\pythoncode{scripts/generated_scripts/pymechanical_script_6.py}

Execute a Command Block:

\pythoncode{scripts/generated_scripts/pymechanical_script_7.py}

Execute a Python Script File :

\pythoncode{scripts/generated_scripts/pymechanical_script_8.py}

Example: Import a Mechdb File and print the count of bodies

\pythoncode{scripts/generated_scripts/pymechanical_script_9.py}


% Third column
% --------------------------------------------------------------------------------
\vfill
% row 1 col 3
\section{\includegraphics[height=\fontcharht\font`\S]{slash.png}Project Specific Operations}
Save Project and Exit Session :

\pythoncode{scripts/generated_scripts/pymechanical_script_10.py}

\section{\includegraphics[height=\fontcharht\font`\S]{slash.png} # B. PyMechanical to \underline{Embed}  a Mechanical Instance as a Python Object}
% \vspace{5mm} %5mm vertical space
% {\color{orange} \rule{\linewidth}{0.1mm}}
To embed a Mechanical instance

\pythoncode{scripts/generated_scripts/pymechanical_script_11.py}

To extract the global API entry points, that are available from built-in Mechanical scripting, and merge them into your global Python global variables:

\pythoncode{scripts/generated_scripts/pymechanical_script_12.py}

From Python, you can now access : 
 

\pythoncode{scripts/generated_scripts/pymechanical_script_13.py}

Example:Import a Mechdb File and Get body Count

\pythoncode{scripts/generated_scripts/pymechanical_script_14.py}

To Use Logging for Debugging:
\pythoncode{scripts/generated_scripts/pymechanical_script_15.py}

% Add subsection
% This section includes useful links to the documentation.
% Examples: installation, API reference, commands, examples.
% Replace 'name of link' with appropriate display text.

\subsection{References from PyMechanical Documentation}
\begin{itemize}
    \item \href{https://mechanical.docs.pyansys.com/version/stable/getting_started/index.html}{\color{blue}{Getting Started}}
    \item \href{https://mechanical.docs.pyansys.com/version/stable/examples/index.html}{\color{blue}{Examples}}
    \item \href{https://mechanical.docs.pyansys.com/version/stable/api/index.html}{\color{blue}{API Reference}}
    \item \href{https://ansyshelp.ansys.com/account/secured?returnurl=/Views/Secured/corp/v231/en/act_script/act_script.html}{\color{blue}{Scripting in Mechanical}}
\end{itemize}
\end{multicols}

% Footer session of the latex with link to documentation and GitHub page
\vspace{-0.15cm}
\noindent\makebox[\linewidth]{\rule{\paperwidth}{4pt}}
\begin{center}
Getting Started with PyMechanical \includegraphics[height=\fontcharht\font`\S]{slash.png} \href{https://github.com/ansys/pymechanical}{\color{blue}{PyMechanical on GitHub}}} \includegraphics[height=\fontcharht\font`\S]{slash.png} Visit \code{\href{https://mechanical.docs.pyansys.com/}}{\color{blue}{mechanical.docs.pyansys.com}}
\end{center}
\end{document}
