\documentclass[9pt,landscape]{article}
\usepackage{{./../static/style}}
\pdfinfo{
  /Title (PyDPF-Post Cheat Sheet)
  /Creator (TeX)
  /Producer (pdfTeX 1.40.0)
  /Author (Ansys)
  /Subject (PyDPF)
  /Keywords (PyAnsys, PyDPF, DPF, Post, Cheat sheet, template)}

\begin{document}
\raggedright
\footnotesize

% Add the title of cheat sheet here
% ----------------------------------------
\begin{center}
     \Huge{\textbf{PyDPF-Post Cheat sheet}} \\
\end{center}

\begin{center}
  \small{\textbf{based on version:0.5 (stable) - API may be subjected to change}} \\
\end{center}

\AddToShipoutPicture*
  {\put(670,577.5){\includegraphics[height = 1.2cm]{ansys.png}}}
\AddToShipoutPictureBG*{\includegraphics[width=\paperwidth]{bground.png}}
\vspace{-0.15cm}
\noindent\makebox[\linewidth]{\rule{\paperwidth}{2pt}}

\begin{multicols}{3}
\setlength{\premulticols}{1pt}
\setlength{\postmulticols}{1pt}
\setlength{\multicolsep}{1pt}
\setlength{\columnsep}{2pt}

% session starts here. 
% First colomn
% --------------------------------------------------------------------------------
\vfill
\section{\includegraphics[height=\fontcharht\font`\S]{slash.png} Load Result File}

Simulation object allows access to the result. The Simulation object must be instantiated with the path for the result file.

\vspace{2mm} %5mm vertical space
% {\color{orange} \rule{\linewidth}{0.1mm}}

(a). On Windows, you can load the result file with:

\pythoncode{scripts/generated_scripts/pydpf-post_script_0.py}

(b). On Linux, you can load the result file with:

\pythoncode{scripts/generated_scripts/pydpf-post_script_1.py}

% row 2 col 1
\section{\includegraphics[height=\fontcharht\font`\S]{slash.png} Fetching Result Metadata}

You can fetch Metadata information from Simulation object, such as the analysis type and the available results.

\pythoncode{scripts/generated_scripts/pydpf-post_script_2.py}

\section{\includegraphics[height=\fontcharht\font`\S]{slash.png} Access Mesh}

From the Simulation object, you can also access the mesh:

\pythoncode{scripts/generated_scripts/pydpf-post_script_3.py}


\section{\includegraphics[height=\fontcharht\font`\S]{slash.png} Access Results}

Using Simulation object, you can access to the results data. You can query results using dedicated methods.
For example, to get the displacement result as,

\pythoncode{scripts/generated_scripts/pydpf-post_script_4.py}

Similarly, you can extract the Stress, Strain and other results quantities as,

\pythoncode{scripts/generated_scripts/pydpf-post_script_5.py}

% Second column
% --------------------------------------------------------------------------------
% row 1 col 2

\vfill

\section{\includegraphics[height=\fontcharht\font`\S]{slash.png} Enable Auto-completion Results Quantities}

You can use physics-oriented API for postprocessing which enable auto-completion of results quantities.

\vspace{2mm} %5mm vertical space

(a). To postprocess Static Simulation results:

\pythoncode{scripts/generated_scripts/pydpf-post_script_6.py}

(b). To postprocess Modal Simulation results:

\pythoncode{scripts/generated_scripts/pydpf-post_script_7.py}

(c). To postprocess Transient Simulation results with animation:

\pythoncode{scripts/generated_scripts/pydpf-post_script_8.py}

(d). To postprocess Harmonic Simulation results:

\pythoncode{scripts/generated_scripts/pydpf-post_script_9.py}

\section{\includegraphics[height=\fontcharht\font`\S]{slash.png} Plot Results}

You can plot the results, for example, total deformation (norm of the displacement vector field) with: 

\pythoncode{scripts/generated_scripts/pydpf-post_script_10.py}

% Third column
% --------------------------------------------------------------------------------
% \vfill
% row 1 col 3

\section{\includegraphics[height=\fontcharht\font`\S]{slash.png} Create and Manipulate a DPF Dataframe}
% \vspace{5mm} %5mm vertical space
% {\color{orange} \rule{\linewidth}{0.1mm}}
You can generate a Dataframe by extracting a result from a simulation, which can be manipulated and viewed differently.

\pythoncode{scripts/generated_scripts/pydpf-post_script_11.py}

The Dataframe's column labels are defined in \underline{Dataframe.columns} as,

\pythoncode{scripts/generated_scripts/pydpf-post_script_12.py}

A ResultIndex index defines the result stored in the Dataframe. You can also check values available for an Index.

\pythoncode{scripts/generated_scripts/pydpf-post_script_13.py}

You can change the number of data rows or columns displayed with:

\pythoncode{scripts/generated_scripts/pydpf-post_script_14.py}

To select specific columns or rows, use the index names as arguments for the \underline{DataFrame.select} method, taking lists of values:

\pythoncode{scripts/generated_scripts/pydpf-post_script_15.py}

You can extract specific data as an array contains in Dataframe with:

\pythoncode{scripts/generated_scripts/pydpf-post_script_16.py}

% Add subsection
% This section includes useful links to the documentation.
% Examples: installation, API reference, commands, examples.
% Replace 'name of link' with appropriate display text.

\subsection{References from PyDPF-Post Documentation}
\begin{multicols}{2}
\begin{itemize}
    \item \href{https://post.docs.pyansys.com/version/stable/getting_started/index.html}{\color{blue}{Getting Started}}
    \item \href{https://post.docs.pyansys.com/version/stable/user_guide/index.html}{\color{blue}{User Guide}}
    \item \href{https://post.docs.pyansys.com/version/stable/examples/index.html}{\color{blue}{Examples}}
    \item \href{https://post.docs.pyansys.com/version/stable/api/index.html}{\color{blue}{API Reference}}
\end{itemize}
\end{multicols}
\end{multicols}

% Footer session of the latex with link to documentation and GitHub page
\vspace{-0.15cm}
\noindent\makebox[\linewidth]{\rule{\paperwidth}{4pt}}
\begin{center}
Getting Started with PyDPF-Post \includegraphics[height=\fontcharht\font`\S]{slash.png} \href{https://github.com/ansys/pydpf-post}{\color{blue}{PyDPF-Post on GitHub}}}
Visit \includegraphics[height=\fontcharht\font`\S]{slash.png} \href{https://post.docs.pyansys.com/version/stable/}{\color{blue}{post.docs.pyansys.com}}} 
\end{center}
\end{document}
